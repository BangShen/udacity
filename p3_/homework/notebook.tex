
% Default to the notebook output style

    


% Inherit from the specified cell style.




    
\documentclass[11pt]{article}

    
    
    \usepackage[T1]{fontenc}
    % Nicer default font (+ math font) than Computer Modern for most use cases
    \usepackage{mathpazo}

    % Basic figure setup, for now with no caption control since it's done
    % automatically by Pandoc (which extracts ![](path) syntax from Markdown).
    \usepackage{graphicx}
    % We will generate all images so they have a width \maxwidth. This means
    % that they will get their normal width if they fit onto the page, but
    % are scaled down if they would overflow the margins.
    \makeatletter
    \def\maxwidth{\ifdim\Gin@nat@width>\linewidth\linewidth
    \else\Gin@nat@width\fi}
    \makeatother
    \let\Oldincludegraphics\includegraphics
    % Set max figure width to be 80% of text width, for now hardcoded.
    \renewcommand{\includegraphics}[1]{\Oldincludegraphics[width=.8\maxwidth]{#1}}
    % Ensure that by default, figures have no caption (until we provide a
    % proper Figure object with a Caption API and a way to capture that
    % in the conversion process - todo).
    \usepackage{caption}
    \DeclareCaptionLabelFormat{nolabel}{}
    \captionsetup{labelformat=nolabel}

    \usepackage{adjustbox} % Used to constrain images to a maximum size 
    \usepackage{xcolor} % Allow colors to be defined
    \usepackage{enumerate} % Needed for markdown enumerations to work
    \usepackage{geometry} % Used to adjust the document margins
    \usepackage{amsmath} % Equations
    \usepackage{amssymb} % Equations
    \usepackage{textcomp} % defines textquotesingle
    % Hack from http://tex.stackexchange.com/a/47451/13684:
    \AtBeginDocument{%
        \def\PYZsq{\textquotesingle}% Upright quotes in Pygmentized code
    }
    \usepackage{upquote} % Upright quotes for verbatim code
    \usepackage{eurosym} % defines \euro
    \usepackage[mathletters]{ucs} % Extended unicode (utf-8) support
    \usepackage[utf8x]{inputenc} % Allow utf-8 characters in the tex document
    \usepackage{fancyvrb} % verbatim replacement that allows latex
    \usepackage{grffile} % extends the file name processing of package graphics 
                         % to support a larger range 
    % The hyperref package gives us a pdf with properly built
    % internal navigation ('pdf bookmarks' for the table of contents,
    % internal cross-reference links, web links for URLs, etc.)
    \usepackage{hyperref}
    \usepackage{longtable} % longtable support required by pandoc >1.10
    \usepackage{booktabs}  % table support for pandoc > 1.12.2
    \usepackage[inline]{enumitem} % IRkernel/repr support (it uses the enumerate* environment)
    \usepackage[normalem]{ulem} % ulem is needed to support strikethroughs (\sout)
                                % normalem makes italics be italics, not underlines
    

    
    
    % Colors for the hyperref package
    \definecolor{urlcolor}{rgb}{0,.145,.698}
    \definecolor{linkcolor}{rgb}{.71,0.21,0.01}
    \definecolor{citecolor}{rgb}{.12,.54,.11}

    % ANSI colors
    \definecolor{ansi-black}{HTML}{3E424D}
    \definecolor{ansi-black-intense}{HTML}{282C36}
    \definecolor{ansi-red}{HTML}{E75C58}
    \definecolor{ansi-red-intense}{HTML}{B22B31}
    \definecolor{ansi-green}{HTML}{00A250}
    \definecolor{ansi-green-intense}{HTML}{007427}
    \definecolor{ansi-yellow}{HTML}{DDB62B}
    \definecolor{ansi-yellow-intense}{HTML}{B27D12}
    \definecolor{ansi-blue}{HTML}{208FFB}
    \definecolor{ansi-blue-intense}{HTML}{0065CA}
    \definecolor{ansi-magenta}{HTML}{D160C4}
    \definecolor{ansi-magenta-intense}{HTML}{A03196}
    \definecolor{ansi-cyan}{HTML}{60C6C8}
    \definecolor{ansi-cyan-intense}{HTML}{258F8F}
    \definecolor{ansi-white}{HTML}{C5C1B4}
    \definecolor{ansi-white-intense}{HTML}{A1A6B2}

    % commands and environments needed by pandoc snippets
    % extracted from the output of `pandoc -s`
    \providecommand{\tightlist}{%
      \setlength{\itemsep}{0pt}\setlength{\parskip}{0pt}}
    \DefineVerbatimEnvironment{Highlighting}{Verbatim}{commandchars=\\\{\}}
    % Add ',fontsize=\small' for more characters per line
    \newenvironment{Shaded}{}{}
    \newcommand{\KeywordTok}[1]{\textcolor[rgb]{0.00,0.44,0.13}{\textbf{{#1}}}}
    \newcommand{\DataTypeTok}[1]{\textcolor[rgb]{0.56,0.13,0.00}{{#1}}}
    \newcommand{\DecValTok}[1]{\textcolor[rgb]{0.25,0.63,0.44}{{#1}}}
    \newcommand{\BaseNTok}[1]{\textcolor[rgb]{0.25,0.63,0.44}{{#1}}}
    \newcommand{\FloatTok}[1]{\textcolor[rgb]{0.25,0.63,0.44}{{#1}}}
    \newcommand{\CharTok}[1]{\textcolor[rgb]{0.25,0.44,0.63}{{#1}}}
    \newcommand{\StringTok}[1]{\textcolor[rgb]{0.25,0.44,0.63}{{#1}}}
    \newcommand{\CommentTok}[1]{\textcolor[rgb]{0.38,0.63,0.69}{\textit{{#1}}}}
    \newcommand{\OtherTok}[1]{\textcolor[rgb]{0.00,0.44,0.13}{{#1}}}
    \newcommand{\AlertTok}[1]{\textcolor[rgb]{1.00,0.00,0.00}{\textbf{{#1}}}}
    \newcommand{\FunctionTok}[1]{\textcolor[rgb]{0.02,0.16,0.49}{{#1}}}
    \newcommand{\RegionMarkerTok}[1]{{#1}}
    \newcommand{\ErrorTok}[1]{\textcolor[rgb]{1.00,0.00,0.00}{\textbf{{#1}}}}
    \newcommand{\NormalTok}[1]{{#1}}
    
    % Additional commands for more recent versions of Pandoc
    \newcommand{\ConstantTok}[1]{\textcolor[rgb]{0.53,0.00,0.00}{{#1}}}
    \newcommand{\SpecialCharTok}[1]{\textcolor[rgb]{0.25,0.44,0.63}{{#1}}}
    \newcommand{\VerbatimStringTok}[1]{\textcolor[rgb]{0.25,0.44,0.63}{{#1}}}
    \newcommand{\SpecialStringTok}[1]{\textcolor[rgb]{0.73,0.40,0.53}{{#1}}}
    \newcommand{\ImportTok}[1]{{#1}}
    \newcommand{\DocumentationTok}[1]{\textcolor[rgb]{0.73,0.13,0.13}{\textit{{#1}}}}
    \newcommand{\AnnotationTok}[1]{\textcolor[rgb]{0.38,0.63,0.69}{\textbf{\textit{{#1}}}}}
    \newcommand{\CommentVarTok}[1]{\textcolor[rgb]{0.38,0.63,0.69}{\textbf{\textit{{#1}}}}}
    \newcommand{\VariableTok}[1]{\textcolor[rgb]{0.10,0.09,0.49}{{#1}}}
    \newcommand{\ControlFlowTok}[1]{\textcolor[rgb]{0.00,0.44,0.13}{\textbf{{#1}}}}
    \newcommand{\OperatorTok}[1]{\textcolor[rgb]{0.40,0.40,0.40}{{#1}}}
    \newcommand{\BuiltInTok}[1]{{#1}}
    \newcommand{\ExtensionTok}[1]{{#1}}
    \newcommand{\PreprocessorTok}[1]{\textcolor[rgb]{0.74,0.48,0.00}{{#1}}}
    \newcommand{\AttributeTok}[1]{\textcolor[rgb]{0.49,0.56,0.16}{{#1}}}
    \newcommand{\InformationTok}[1]{\textcolor[rgb]{0.38,0.63,0.69}{\textbf{\textit{{#1}}}}}
    \newcommand{\WarningTok}[1]{\textcolor[rgb]{0.38,0.63,0.69}{\textbf{\textit{{#1}}}}}
    
    
    % Define a nice break command that doesn't care if a line doesn't already
    % exist.
    \def\br{\hspace*{\fill} \\* }
    % Math Jax compatability definitions
    \def\gt{>}
    \def\lt{<}
    % Document parameters
    \title{Report}
    
    
    

    % Pygments definitions
    
\makeatletter
\def\PY@reset{\let\PY@it=\relax \let\PY@bf=\relax%
    \let\PY@ul=\relax \let\PY@tc=\relax%
    \let\PY@bc=\relax \let\PY@ff=\relax}
\def\PY@tok#1{\csname PY@tok@#1\endcsname}
\def\PY@toks#1+{\ifx\relax#1\empty\else%
    \PY@tok{#1}\expandafter\PY@toks\fi}
\def\PY@do#1{\PY@bc{\PY@tc{\PY@ul{%
    \PY@it{\PY@bf{\PY@ff{#1}}}}}}}
\def\PY#1#2{\PY@reset\PY@toks#1+\relax+\PY@do{#2}}

\expandafter\def\csname PY@tok@w\endcsname{\def\PY@tc##1{\textcolor[rgb]{0.73,0.73,0.73}{##1}}}
\expandafter\def\csname PY@tok@c\endcsname{\let\PY@it=\textit\def\PY@tc##1{\textcolor[rgb]{0.25,0.50,0.50}{##1}}}
\expandafter\def\csname PY@tok@cp\endcsname{\def\PY@tc##1{\textcolor[rgb]{0.74,0.48,0.00}{##1}}}
\expandafter\def\csname PY@tok@k\endcsname{\let\PY@bf=\textbf\def\PY@tc##1{\textcolor[rgb]{0.00,0.50,0.00}{##1}}}
\expandafter\def\csname PY@tok@kp\endcsname{\def\PY@tc##1{\textcolor[rgb]{0.00,0.50,0.00}{##1}}}
\expandafter\def\csname PY@tok@kt\endcsname{\def\PY@tc##1{\textcolor[rgb]{0.69,0.00,0.25}{##1}}}
\expandafter\def\csname PY@tok@o\endcsname{\def\PY@tc##1{\textcolor[rgb]{0.40,0.40,0.40}{##1}}}
\expandafter\def\csname PY@tok@ow\endcsname{\let\PY@bf=\textbf\def\PY@tc##1{\textcolor[rgb]{0.67,0.13,1.00}{##1}}}
\expandafter\def\csname PY@tok@nb\endcsname{\def\PY@tc##1{\textcolor[rgb]{0.00,0.50,0.00}{##1}}}
\expandafter\def\csname PY@tok@nf\endcsname{\def\PY@tc##1{\textcolor[rgb]{0.00,0.00,1.00}{##1}}}
\expandafter\def\csname PY@tok@nc\endcsname{\let\PY@bf=\textbf\def\PY@tc##1{\textcolor[rgb]{0.00,0.00,1.00}{##1}}}
\expandafter\def\csname PY@tok@nn\endcsname{\let\PY@bf=\textbf\def\PY@tc##1{\textcolor[rgb]{0.00,0.00,1.00}{##1}}}
\expandafter\def\csname PY@tok@ne\endcsname{\let\PY@bf=\textbf\def\PY@tc##1{\textcolor[rgb]{0.82,0.25,0.23}{##1}}}
\expandafter\def\csname PY@tok@nv\endcsname{\def\PY@tc##1{\textcolor[rgb]{0.10,0.09,0.49}{##1}}}
\expandafter\def\csname PY@tok@no\endcsname{\def\PY@tc##1{\textcolor[rgb]{0.53,0.00,0.00}{##1}}}
\expandafter\def\csname PY@tok@nl\endcsname{\def\PY@tc##1{\textcolor[rgb]{0.63,0.63,0.00}{##1}}}
\expandafter\def\csname PY@tok@ni\endcsname{\let\PY@bf=\textbf\def\PY@tc##1{\textcolor[rgb]{0.60,0.60,0.60}{##1}}}
\expandafter\def\csname PY@tok@na\endcsname{\def\PY@tc##1{\textcolor[rgb]{0.49,0.56,0.16}{##1}}}
\expandafter\def\csname PY@tok@nt\endcsname{\let\PY@bf=\textbf\def\PY@tc##1{\textcolor[rgb]{0.00,0.50,0.00}{##1}}}
\expandafter\def\csname PY@tok@nd\endcsname{\def\PY@tc##1{\textcolor[rgb]{0.67,0.13,1.00}{##1}}}
\expandafter\def\csname PY@tok@s\endcsname{\def\PY@tc##1{\textcolor[rgb]{0.73,0.13,0.13}{##1}}}
\expandafter\def\csname PY@tok@sd\endcsname{\let\PY@it=\textit\def\PY@tc##1{\textcolor[rgb]{0.73,0.13,0.13}{##1}}}
\expandafter\def\csname PY@tok@si\endcsname{\let\PY@bf=\textbf\def\PY@tc##1{\textcolor[rgb]{0.73,0.40,0.53}{##1}}}
\expandafter\def\csname PY@tok@se\endcsname{\let\PY@bf=\textbf\def\PY@tc##1{\textcolor[rgb]{0.73,0.40,0.13}{##1}}}
\expandafter\def\csname PY@tok@sr\endcsname{\def\PY@tc##1{\textcolor[rgb]{0.73,0.40,0.53}{##1}}}
\expandafter\def\csname PY@tok@ss\endcsname{\def\PY@tc##1{\textcolor[rgb]{0.10,0.09,0.49}{##1}}}
\expandafter\def\csname PY@tok@sx\endcsname{\def\PY@tc##1{\textcolor[rgb]{0.00,0.50,0.00}{##1}}}
\expandafter\def\csname PY@tok@m\endcsname{\def\PY@tc##1{\textcolor[rgb]{0.40,0.40,0.40}{##1}}}
\expandafter\def\csname PY@tok@gh\endcsname{\let\PY@bf=\textbf\def\PY@tc##1{\textcolor[rgb]{0.00,0.00,0.50}{##1}}}
\expandafter\def\csname PY@tok@gu\endcsname{\let\PY@bf=\textbf\def\PY@tc##1{\textcolor[rgb]{0.50,0.00,0.50}{##1}}}
\expandafter\def\csname PY@tok@gd\endcsname{\def\PY@tc##1{\textcolor[rgb]{0.63,0.00,0.00}{##1}}}
\expandafter\def\csname PY@tok@gi\endcsname{\def\PY@tc##1{\textcolor[rgb]{0.00,0.63,0.00}{##1}}}
\expandafter\def\csname PY@tok@gr\endcsname{\def\PY@tc##1{\textcolor[rgb]{1.00,0.00,0.00}{##1}}}
\expandafter\def\csname PY@tok@ge\endcsname{\let\PY@it=\textit}
\expandafter\def\csname PY@tok@gs\endcsname{\let\PY@bf=\textbf}
\expandafter\def\csname PY@tok@gp\endcsname{\let\PY@bf=\textbf\def\PY@tc##1{\textcolor[rgb]{0.00,0.00,0.50}{##1}}}
\expandafter\def\csname PY@tok@go\endcsname{\def\PY@tc##1{\textcolor[rgb]{0.53,0.53,0.53}{##1}}}
\expandafter\def\csname PY@tok@gt\endcsname{\def\PY@tc##1{\textcolor[rgb]{0.00,0.27,0.87}{##1}}}
\expandafter\def\csname PY@tok@err\endcsname{\def\PY@bc##1{\setlength{\fboxsep}{0pt}\fcolorbox[rgb]{1.00,0.00,0.00}{1,1,1}{\strut ##1}}}
\expandafter\def\csname PY@tok@kc\endcsname{\let\PY@bf=\textbf\def\PY@tc##1{\textcolor[rgb]{0.00,0.50,0.00}{##1}}}
\expandafter\def\csname PY@tok@kd\endcsname{\let\PY@bf=\textbf\def\PY@tc##1{\textcolor[rgb]{0.00,0.50,0.00}{##1}}}
\expandafter\def\csname PY@tok@kn\endcsname{\let\PY@bf=\textbf\def\PY@tc##1{\textcolor[rgb]{0.00,0.50,0.00}{##1}}}
\expandafter\def\csname PY@tok@kr\endcsname{\let\PY@bf=\textbf\def\PY@tc##1{\textcolor[rgb]{0.00,0.50,0.00}{##1}}}
\expandafter\def\csname PY@tok@bp\endcsname{\def\PY@tc##1{\textcolor[rgb]{0.00,0.50,0.00}{##1}}}
\expandafter\def\csname PY@tok@fm\endcsname{\def\PY@tc##1{\textcolor[rgb]{0.00,0.00,1.00}{##1}}}
\expandafter\def\csname PY@tok@vc\endcsname{\def\PY@tc##1{\textcolor[rgb]{0.10,0.09,0.49}{##1}}}
\expandafter\def\csname PY@tok@vg\endcsname{\def\PY@tc##1{\textcolor[rgb]{0.10,0.09,0.49}{##1}}}
\expandafter\def\csname PY@tok@vi\endcsname{\def\PY@tc##1{\textcolor[rgb]{0.10,0.09,0.49}{##1}}}
\expandafter\def\csname PY@tok@vm\endcsname{\def\PY@tc##1{\textcolor[rgb]{0.10,0.09,0.49}{##1}}}
\expandafter\def\csname PY@tok@sa\endcsname{\def\PY@tc##1{\textcolor[rgb]{0.73,0.13,0.13}{##1}}}
\expandafter\def\csname PY@tok@sb\endcsname{\def\PY@tc##1{\textcolor[rgb]{0.73,0.13,0.13}{##1}}}
\expandafter\def\csname PY@tok@sc\endcsname{\def\PY@tc##1{\textcolor[rgb]{0.73,0.13,0.13}{##1}}}
\expandafter\def\csname PY@tok@dl\endcsname{\def\PY@tc##1{\textcolor[rgb]{0.73,0.13,0.13}{##1}}}
\expandafter\def\csname PY@tok@s2\endcsname{\def\PY@tc##1{\textcolor[rgb]{0.73,0.13,0.13}{##1}}}
\expandafter\def\csname PY@tok@sh\endcsname{\def\PY@tc##1{\textcolor[rgb]{0.73,0.13,0.13}{##1}}}
\expandafter\def\csname PY@tok@s1\endcsname{\def\PY@tc##1{\textcolor[rgb]{0.73,0.13,0.13}{##1}}}
\expandafter\def\csname PY@tok@mb\endcsname{\def\PY@tc##1{\textcolor[rgb]{0.40,0.40,0.40}{##1}}}
\expandafter\def\csname PY@tok@mf\endcsname{\def\PY@tc##1{\textcolor[rgb]{0.40,0.40,0.40}{##1}}}
\expandafter\def\csname PY@tok@mh\endcsname{\def\PY@tc##1{\textcolor[rgb]{0.40,0.40,0.40}{##1}}}
\expandafter\def\csname PY@tok@mi\endcsname{\def\PY@tc##1{\textcolor[rgb]{0.40,0.40,0.40}{##1}}}
\expandafter\def\csname PY@tok@il\endcsname{\def\PY@tc##1{\textcolor[rgb]{0.40,0.40,0.40}{##1}}}
\expandafter\def\csname PY@tok@mo\endcsname{\def\PY@tc##1{\textcolor[rgb]{0.40,0.40,0.40}{##1}}}
\expandafter\def\csname PY@tok@ch\endcsname{\let\PY@it=\textit\def\PY@tc##1{\textcolor[rgb]{0.25,0.50,0.50}{##1}}}
\expandafter\def\csname PY@tok@cm\endcsname{\let\PY@it=\textit\def\PY@tc##1{\textcolor[rgb]{0.25,0.50,0.50}{##1}}}
\expandafter\def\csname PY@tok@cpf\endcsname{\let\PY@it=\textit\def\PY@tc##1{\textcolor[rgb]{0.25,0.50,0.50}{##1}}}
\expandafter\def\csname PY@tok@c1\endcsname{\let\PY@it=\textit\def\PY@tc##1{\textcolor[rgb]{0.25,0.50,0.50}{##1}}}
\expandafter\def\csname PY@tok@cs\endcsname{\let\PY@it=\textit\def\PY@tc##1{\textcolor[rgb]{0.25,0.50,0.50}{##1}}}

\def\PYZbs{\char`\\}
\def\PYZus{\char`\_}
\def\PYZob{\char`\{}
\def\PYZcb{\char`\}}
\def\PYZca{\char`\^}
\def\PYZam{\char`\&}
\def\PYZlt{\char`\<}
\def\PYZgt{\char`\>}
\def\PYZsh{\char`\#}
\def\PYZpc{\char`\%}
\def\PYZdl{\char`\$}
\def\PYZhy{\char`\-}
\def\PYZsq{\char`\'}
\def\PYZdq{\char`\"}
\def\PYZti{\char`\~}
% for compatibility with earlier versions
\def\PYZat{@}
\def\PYZlb{[}
\def\PYZrb{]}
\makeatother


    % Exact colors from NB
    \definecolor{incolor}{rgb}{0.0, 0.0, 0.5}
    \definecolor{outcolor}{rgb}{0.545, 0.0, 0.0}



    
    % Prevent overflowing lines due to hard-to-break entities
    \sloppy 
    % Setup hyperref package
    \hypersetup{
      breaklinks=true,  % so long urls are correctly broken across lines
      colorlinks=true,
      urlcolor=urlcolor,
      linkcolor=linkcolor,
      citecolor=citecolor,
      }
    % Slightly bigger margins than the latex defaults
    
    \geometry{verbose,tmargin=1in,bmargin=1in,lmargin=1in,rmargin=1in}
    
    

    \begin{document}
    
    
    \maketitle
    
    

    
    \#

Report on OpenStreetMap Data Cleaning Project

\section{Contents}\label{contents}

\begin{itemize}
\tightlist
\item
  Map area
\item
  Data Overview

  \begin{itemize}
  \tightlist
  \item
    ** \emph{Files size} **: know the size of the selected file
  \item
    ** \emph{Tags number} **: know how many tags in the files
  \end{itemize}
\item
  Data cleaning

  \begin{itemize}
  \tightlist
  \item
    \textbf{Main problems of data}
  \item
    \textbf{Results after updating}
  \end{itemize}
\item
  Data Exploration from sql data

  \begin{itemize}
  \tightlist
  \item
    \textbf{Size of files}
  \item
    \textbf{Top 10 contributing users of Beijing map}
  \item
    \textbf{Hutong numbers in Beijing}
  \item
    \textbf{Source distribution}
  \end{itemize}
\item
  Problems encountered and solutions
\item
  Conclusions
\end{itemize}

    \section{Map area}\label{map-area}

    \textbf{Beijing, China}

Beijing is the capital of China with a lot of acient buildings and
modern good-designed architectures. I would like to explore some
information about our great capital. The link of this file is:
\href{https://mapzen.com/data/metro-extracts/metro/beijing_china/}{link}

    \section{Data Overview}\label{data-overview}

    \begin{itemize}
\tightlist
\item
  \textbf{Files Size} The size of the file is \textbf{164 megabytes},
  this is qualified with the requirement of udacity course
\end{itemize}

    \begin{Verbatim}[commandchars=\\\{\}]
{\color{incolor}In [{\color{incolor}60}]:} \PY{k+kn}{import} \PY{n+nn}{os}
         \PY{n+nb}{print} \PY{p}{(}\PY{l+s+s1}{\PYZsq{}}\PY{l+s+s1}{beijing\PYZus{}china.osm:}\PY{l+s+si}{\PYZob{}\PYZcb{}}\PY{l+s+s1}{ MB}\PY{l+s+s1}{\PYZsq{}}\PY{o}{.}\PY{n}{format}\PY{p}{(}\PY{n}{os}\PY{o}{.}\PY{n}{path}\PY{o}{.}\PY{n}{getsize}\PY{p}{(}\PY{l+s+s1}{\PYZsq{}}\PY{l+s+s1}{beijing\PYZus{}china.osm}\PY{l+s+s1}{\PYZsq{}}\PY{p}{)} \PY{o}{\PYZgt{}\PYZgt{}} \PY{l+m+mi}{20}\PY{p}{)}\PY{p}{)}
\end{Verbatim}


    \begin{Verbatim}[commandchars=\\\{\}]
beijing\_china.osm:164 MB

    \end{Verbatim}

    \begin{itemize}
\tightlist
\item
  \textbf{Tags number} Before cleanig data, I want to have a overview of
  data that I have, the first thing I wanna check is tags number, and
  which is counting by this function named count\_tag()
  \textbf{Outputs:} \textbf{\{'osm': 1, 'bounds': 1, 'node': 777694,
  'tag': 339853, 'way': 115041, 'nd': 928128, 'relation': 5601,
  'member': 60922\}}
\end{itemize}

    \begin{Verbatim}[commandchars=\\\{\}]
{\color{incolor}In [{\color{incolor} }]:} \PY{k}{def} \PY{n+nf}{count\PYZus{}tags}\PY{p}{(}\PY{n}{filename}\PY{p}{)}\PY{p}{:}
            \PY{n}{tags} \PY{o}{=} \PY{p}{\PYZob{}}\PY{p}{\PYZcb{}}
            \PY{n}{taglist} \PY{o}{=} \PY{p}{[}\PY{p}{]}
            \PY{k}{for} \PY{n}{\PYZus{}}\PY{p}{,}\PY{n}{elem} \PY{o+ow}{in} \PY{n}{ET}\PY{o}{.}\PY{n}{iterparse}\PY{p}{(}\PY{n}{filename}\PY{p}{,}\PY{n}{events}\PY{o}{=}\PY{p}{(}\PY{l+s+s1}{\PYZsq{}}\PY{l+s+s1}{start}\PY{l+s+s1}{\PYZsq{}}\PY{p}{,}\PY{p}{)}\PY{p}{)}\PY{p}{:}
                \PY{n}{taglist}\PY{o}{.}\PY{n}{append}\PY{p}{(}\PY{n}{elem}\PY{o}{.}\PY{n}{tag}\PY{p}{)}
            \PY{k}{for} \PY{n}{tag} \PY{o+ow}{in} \PY{n}{taglist}\PY{p}{:}
                \PY{k}{if} \PY{n}{tag} \PY{o+ow}{not} \PY{o+ow}{in} \PY{n}{tags}\PY{p}{:}
                    \PY{n}{tags}\PY{p}{[}\PY{n}{tag}\PY{p}{]} \PY{o}{=} \PY{l+m+mi}{1}
                \PY{k}{else}\PY{p}{:}
                    \PY{n}{tags}\PY{p}{[}\PY{n}{tag}\PY{p}{]} \PY{o}{+}\PY{o}{=} \PY{l+m+mi}{1}
            \PY{k}{return} \PY{n}{tags}
\end{Verbatim}


    \section{Data cleaning process}\label{data-cleaning-process}

\subsubsection{1. Main problems of data}\label{main-problems-of-data}

\begin{itemize}
\tightlist
\item
  \textbf{Way and node names}: There are a lot of way names, most of are
  in Chinese, but some are in English and Pinyin, such as
  \textbf{Rongxian Hutong and Embassy of Germany} , those names need to
  be converted into Chinese characters for further exploration.
\item
  \textbf{Phone format problems}: some phone is written as
  '01087671788',but another is writen as '+86 010 69618888'
\item
  \textbf{Node sourse}: samples node sourse is like '\{'Bing', 'GPX',
  'bing', 'gps','Bing, 2005-04'\}', which is not
  unified,这个可以作为后面的other ideas 来处理
\item
  \textbf{Cuisine}: samples cuisine is like '\{'american;burger',
  'chinese', 'german'\}',which will changed to country name
\end{itemize}

    \subsubsection{2.Data cleaning function}\label{data-cleaning-function}

The function used in the part is shown as below, names of nodes and ways
in pinyin will be repalce into Chinese charactes, such as 'Rongxian
Hutong' will be changed into 'Rongxian 胡同', translating pinyin into
Chinese is tricky, so I did nothing with that, and actually, I pay more
attention to the type of a node or way(such as '胡同' or '公路') instead
of it's special name.

    \begin{Verbatim}[commandchars=\\\{\}]
{\color{incolor}In [{\color{incolor}1}]:} \PY{c+c1}{\PYZsh{} functions for updating cuisine phone and source of ways or nodes}
        \PY{k}{def} \PY{n+nf}{mapping}\PY{p}{(}\PY{n}{name}\PY{p}{,}\PY{n}{mapping\PYZus{}dict}\PY{p}{)}\PY{p}{:}
            \PY{n}{name\PYZus{}list} \PY{o}{=} \PY{n}{name}\PY{o}{.}\PY{n}{split}\PY{p}{(}\PY{p}{)}
            \PY{k}{if} \PY{n}{name\PYZus{}list}\PY{p}{[}\PY{o}{\PYZhy{}}\PY{l+m+mi}{1}\PY{p}{]} \PY{o+ow}{in} \PY{n}{mapping\PYZus{}dict}\PY{p}{:}
                \PY{n}{name\PYZus{}list}\PY{p}{[}\PY{o}{\PYZhy{}}\PY{l+m+mi}{1}\PY{p}{]} \PY{o}{=} \PY{n}{mapping\PYZus{}dict}\PY{p}{[}\PY{n}{name\PYZus{}list}\PY{p}{[}\PY{o}{\PYZhy{}}\PY{l+m+mi}{1}\PY{p}{]}\PY{p}{]}
                \PY{k}{return} \PY{l+s+s1}{\PYZsq{}}\PY{l+s+s1}{ }\PY{l+s+s1}{\PYZsq{}}\PY{o}{.}\PY{n}{join}\PY{p}{(}\PY{n}{name\PYZus{}list}\PY{p}{)}
        \PY{n}{mapping\PYZus{}dict} \PY{o}{=} \PY{p}{\PYZob{}}
            \PY{l+s+s1}{\PYZsq{}}\PY{l+s+s1}{Road}\PY{l+s+s1}{\PYZsq{}}\PY{p}{:}\PY{l+s+s1}{\PYZsq{}}\PY{l+s+s1}{路}\PY{l+s+s1}{\PYZsq{}}\PY{p}{,}
            \PY{l+s+s1}{\PYZsq{}}\PY{l+s+s1}{Expressway}\PY{l+s+s1}{\PYZsq{}}\PY{p}{:}\PY{l+s+s1}{\PYZsq{}}\PY{l+s+s1}{高速公路}\PY{l+s+s1}{\PYZsq{}}\PY{p}{,}
            \PY{l+s+s1}{\PYZsq{}}\PY{l+s+s1}{Lu}\PY{l+s+s1}{\PYZsq{}}\PY{p}{:}\PY{l+s+s1}{\PYZsq{}}\PY{l+s+s1}{路}\PY{l+s+s1}{\PYZsq{}}\PY{p}{,}
            \PY{l+s+s1}{\PYZsq{}}\PY{l+s+s1}{lu}\PY{l+s+s1}{\PYZsq{}}\PY{p}{:}\PY{l+s+s1}{\PYZsq{}}\PY{l+s+s1}{路}\PY{l+s+s1}{\PYZsq{}}\PY{p}{,}
            \PY{l+s+s1}{\PYZsq{}}\PY{l+s+s1}{Hutong}\PY{l+s+s1}{\PYZsq{}}\PY{p}{:}\PY{l+s+s1}{\PYZsq{}}\PY{l+s+s1}{胡同}\PY{l+s+s1}{\PYZsq{}}\PY{p}{,}
            \PY{l+s+s1}{\PYZsq{}}\PY{l+s+s1}{hutong}\PY{l+s+s1}{\PYZsq{}}\PY{p}{:}\PY{l+s+s1}{\PYZsq{}}\PY{l+s+s1}{胡同}\PY{l+s+s1}{\PYZsq{}}\PY{p}{,}
            \PY{l+s+s1}{\PYZsq{}}\PY{l+s+s1}{Embassy}\PY{l+s+s1}{\PYZsq{}}\PY{p}{:}\PY{l+s+s1}{\PYZsq{}}\PY{l+s+s1}{大使馆}\PY{l+s+s1}{\PYZsq{}}\PY{p}{,}
            \PY{l+s+s1}{\PYZsq{}}\PY{l+s+s1}{Coffee}\PY{l+s+s1}{\PYZsq{}}\PY{p}{:}\PY{l+s+s1}{\PYZsq{}}\PY{l+s+s1}{咖啡厅}\PY{l+s+s1}{\PYZsq{}}
        \PY{p}{\PYZcb{}}
        
        \PY{k}{def} \PY{n+nf}{update\PYZus{}cuisine}\PY{p}{(}\PY{n}{cuisine}\PY{p}{)}\PY{p}{:}
            \PY{k}{return} \PY{n}{cuisine}\PY{o}{.}\PY{n}{split}\PY{p}{(}\PY{l+s+s1}{\PYZsq{}}\PY{l+s+s1}{;}\PY{l+s+s1}{\PYZsq{}}\PY{p}{)}\PY{p}{[}\PY{l+m+mi}{0}\PY{p}{]}
        \PY{k}{def} \PY{n+nf}{update\PYZus{}phone}\PY{p}{(}\PY{n}{phone}\PY{p}{)}\PY{p}{:}
            \PY{k}{if} \PY{n+nb}{len}\PY{p}{(}\PY{n}{phone}\PY{p}{)} \PY{o}{==} \PY{l+m+mi}{8}\PY{p}{:}
                \PY{k}{return} \PY{l+s+s1}{\PYZsq{}}\PY{l+s+s1}{+86010}\PY{l+s+s1}{\PYZsq{}}\PY{o}{+}\PY{n}{phone}
            \PY{k}{elif} \PY{n+nb}{len}\PY{p}{(}\PY{n}{phone}\PY{p}{)} \PY{o}{==} \PY{l+m+mi}{11}\PY{p}{:}
                \PY{k}{return} \PY{l+s+s1}{\PYZsq{}}\PY{l+s+s1}{+86}\PY{l+s+s1}{\PYZsq{}}\PY{o}{+}\PY{n}{phone}
            \PY{k}{elif} \PY{n+nb}{len}\PY{p}{(}\PY{n}{phone}\PY{p}{)} \PY{o}{\PYZgt{}}\PY{l+m+mi}{11}\PY{p}{:}
                \PY{k}{return} \PY{n}{phone}\PY{o}{.}\PY{n}{replace}\PY{p}{(}\PY{l+s+s1}{\PYZsq{}}\PY{l+s+s1}{ }\PY{l+s+s1}{\PYZsq{}}\PY{p}{,}\PY{l+s+s1}{\PYZsq{}}\PY{l+s+s1}{\PYZsq{}}\PY{p}{)}
            \PY{k}{elif} \PY{n+nb}{len}\PY{p}{(}\PY{n}{phone}\PY{p}{)} \PY{o}{\PYZlt{}}\PY{l+m+mi}{8}\PY{p}{:}
                \PY{k}{return} \PY{l+s+s1}{\PYZsq{}}\PY{l+s+s1}{Error}\PY{l+s+s1}{\PYZsq{}}\PY{o}{+}\PY{l+s+s1}{\PYZsq{}}\PY{l+s+s1}{+}\PY{l+s+s1}{\PYZsq{}}\PY{o}{+}\PY{n}{phone}
            \PY{k}{elif} \PY{n+nb}{len}\PY{p}{(}\PY{n}{phone}\PY{p}{)} \PY{o}{\PYZgt{}}\PY{l+m+mi}{8} \PY{o+ow}{and} \PY{n+nb}{len}\PY{p}{(}\PY{n}{phone}\PY{p}{)} \PY{o}{\PYZlt{}} \PY{l+m+mi}{11}\PY{p}{:}
                \PY{k}{return} \PY{l+s+s1}{\PYZsq{}}\PY{l+s+s1}{Error}\PY{l+s+s1}{\PYZsq{}}\PY{o}{+}\PY{l+s+s1}{\PYZsq{}}\PY{l+s+s1}{+}\PY{l+s+s1}{\PYZsq{}}\PY{o}{+}\PY{n}{phone}
        \PY{k}{def} \PY{n+nf}{update\PYZus{}source}\PY{p}{(}\PY{n}{source}\PY{p}{)}\PY{p}{:}
            \PY{k+kn}{import} \PY{n+nn}{re}
            \PY{n}{sourcename} \PY{o}{=} \PY{n}{re}\PY{o}{.}\PY{n}{split}\PY{p}{(}\PY{l+s+sa}{r}\PY{l+s+s1}{\PYZsq{}}\PY{l+s+s1}{[,;. ]}\PY{l+s+s1}{\PYZsq{}}\PY{p}{,}\PY{n}{source}\PY{p}{)}
            \PY{n}{sourcename\PYZus{}0} \PY{o}{=} \PY{n}{sourcename}\PY{p}{[}\PY{l+m+mi}{0}\PY{p}{]}
            \PY{n}{reg\PYZus{}bing} \PY{o}{=} \PY{n}{re}\PY{o}{.}\PY{n}{compile}\PY{p}{(}\PY{n}{re}\PY{o}{.}\PY{n}{escape}\PY{p}{(}\PY{l+s+s1}{\PYZsq{}}\PY{l+s+s1}{bing}\PY{l+s+s1}{\PYZsq{}}\PY{p}{)}\PY{p}{,} \PY{n}{re}\PY{o}{.}\PY{n}{IGNORECASE}\PY{p}{)}
            \PY{n}{reg\PYZus{}gps} \PY{o}{=} \PY{n}{re}\PY{o}{.}\PY{n}{compile}\PY{p}{(}\PY{n}{re}\PY{o}{.}\PY{n}{escape}\PY{p}{(}\PY{l+s+s1}{\PYZsq{}}\PY{l+s+s1}{gps}\PY{l+s+s1}{\PYZsq{}}\PY{p}{)}\PY{p}{,} \PY{n}{re}\PY{o}{.}\PY{n}{IGNORECASE}\PY{p}{)}
            \PY{n}{reg\PYZus{}yahoo} \PY{o}{=} \PY{n}{re}\PY{o}{.}\PY{n}{compile}\PY{p}{(}\PY{n}{re}\PY{o}{.}\PY{n}{escape}\PY{p}{(}\PY{l+s+s1}{\PYZsq{}}\PY{l+s+s1}{yahoo}\PY{l+s+s1}{\PYZsq{}}\PY{p}{)}\PY{p}{,} \PY{n}{re}\PY{o}{.}\PY{n}{IGNORECASE}\PY{p}{)}
            \PY{n}{reg\PYZus{}survey} \PY{o}{=} \PY{n}{re}\PY{o}{.}\PY{n}{compile}\PY{p}{(}\PY{n}{re}\PY{o}{.}\PY{n}{escape}\PY{p}{(}\PY{l+s+s1}{\PYZsq{}}\PY{l+s+s1}{survey}\PY{l+s+s1}{\PYZsq{}}\PY{p}{)}\PY{p}{,} \PY{n}{re}\PY{o}{.}\PY{n}{IGNORECASE}\PY{p}{)}
            \PY{n}{sourcename\PYZus{}0} \PY{o}{=} \PY{n}{reg\PYZus{}bing}\PY{o}{.}\PY{n}{sub}\PY{p}{(}\PY{l+s+s1}{\PYZsq{}}\PY{l+s+s1}{Bing}\PY{l+s+s1}{\PYZsq{}}\PY{p}{,} \PY{n}{sourcename\PYZus{}0}\PY{p}{)}
            \PY{n}{sourcename\PYZus{}0} \PY{o}{=} \PY{n}{reg\PYZus{}gps}\PY{o}{.}\PY{n}{sub}\PY{p}{(}\PY{l+s+s1}{\PYZsq{}}\PY{l+s+s1}{GPS}\PY{l+s+s1}{\PYZsq{}}\PY{p}{,} \PY{n}{sourcename\PYZus{}0}\PY{p}{)}
            \PY{n}{sourcename\PYZus{}0} \PY{o}{=} \PY{n}{reg\PYZus{}yahoo}\PY{o}{.}\PY{n}{sub}\PY{p}{(}\PY{l+s+s1}{\PYZsq{}}\PY{l+s+s1}{Yahoo}\PY{l+s+s1}{\PYZsq{}}\PY{p}{,} \PY{n}{sourcename\PYZus{}0}\PY{p}{)}
            \PY{n}{sourcename\PYZus{}0} \PY{o}{=} \PY{n}{reg\PYZus{}survey}\PY{o}{.}\PY{n}{sub}\PY{p}{(}\PY{l+s+s1}{\PYZsq{}}\PY{l+s+s1}{Survey}\PY{l+s+s1}{\PYZsq{}}\PY{p}{,} \PY{n}{sourcename\PYZus{}0}\PY{p}{)}
            \PY{k}{if} \PY{n}{sourcename\PYZus{}0} \PY{o+ow}{not} \PY{o+ow}{in} \PY{p}{[}\PY{l+s+s1}{\PYZsq{}}\PY{l+s+s1}{Bing}\PY{l+s+s1}{\PYZsq{}}\PY{p}{,}\PY{l+s+s1}{\PYZsq{}}\PY{l+s+s1}{GPS}\PY{l+s+s1}{\PYZsq{}}\PY{p}{,}\PY{l+s+s1}{\PYZsq{}}\PY{l+s+s1}{Yahoo}\PY{l+s+s1}{\PYZsq{}}\PY{p}{,}\PY{l+s+s1}{\PYZsq{}}\PY{l+s+s1}{Survey}\PY{l+s+s1}{\PYZsq{}}\PY{p}{]}\PY{p}{:}
                \PY{n}{sourcename\PYZus{}0} \PY{o}{=} \PY{l+s+s1}{\PYZsq{}}\PY{l+s+s1}{other}\PY{l+s+s1}{\PYZsq{}}
            \PY{k}{return} \PY{n}{sourcename\PYZus{}0}
\end{Verbatim}


    \subsubsection{3.Results after updating}\label{results-after-updating}

The data after cleaning via \textbf{update\_functions} will be saved to
the database: * \textbf{source}:There are lots of sources, I just clean
main source such as bing, gps etc. but some other source like 'local'
and 'map.sogou.com' are also been found, those source can be classified
as 'others'. * \textbf{phone}:The phone format here is more complex than
what I thought, I did not do much work on this since few things can be
done by phone number analysis.

    \begin{Verbatim}[commandchars=\\\{\}]
{\color{incolor}In [{\color{incolor}53}]:} \PY{c+c1}{\PYZsh{} source updating, }
         \PY{k+kn}{import} \PY{n+nn}{sqlite3}
         \PY{n}{conn} \PY{o}{=} \PY{n}{sqlite3}\PY{o}{.}\PY{n}{connect}\PY{p}{(}\PY{l+s+s1}{\PYZsq{}}\PY{l+s+s1}{p3\PYZus{}project.db}\PY{l+s+s1}{\PYZsq{}}\PY{p}{)}
         \PY{n}{c} \PY{o}{=} \PY{n}{conn}\PY{o}{.}\PY{n}{cursor}\PY{p}{(}\PY{p}{)}
         \PY{n}{c}\PY{o}{.}\PY{n}{execute}\PY{p}{(}\PY{l+s+s1}{\PYZsq{}}\PY{l+s+s1}{select value,count(*) as num from ways\PYZus{}tags where key = }\PY{l+s+s1}{\PYZdq{}}\PY{l+s+s1}{source}\PY{l+s+s1}{\PYZdq{}}\PY{l+s+s1}{ group by value order by num desc}\PY{l+s+s1}{\PYZsq{}}\PY{p}{)}\PY{o}{.}\PY{n}{fetchall}\PY{p}{(}\PY{p}{)}
\end{Verbatim}


\begin{Verbatim}[commandchars=\\\{\}]
{\color{outcolor}Out[{\color{outcolor}53}]:} [('Bing', 3319), ('Yahoo', 1027), ('GPS', 371), ('other', 226), ('Survey', 35)]
\end{Verbatim}
            
    \begin{Verbatim}[commandchars=\\\{\}]
{\color{incolor}In [{\color{incolor}55}]:} \PY{c+c1}{\PYZsh{} phone updating}
         \PY{n}{c}\PY{o}{.}\PY{n}{execute}\PY{p}{(}\PY{l+s+s1}{\PYZsq{}}\PY{l+s+s1}{select value as num from nodes\PYZus{}tags where key = }\PY{l+s+s1}{\PYZdq{}}\PY{l+s+s1}{phone}\PY{l+s+s1}{\PYZdq{}}\PY{l+s+s1}{ limit 10}\PY{l+s+s1}{\PYZsq{}}\PY{p}{)}\PY{o}{.}\PY{n}{fetchall}\PY{p}{(}\PY{p}{)}
\end{Verbatim}


\begin{Verbatim}[commandchars=\\\{\}]
{\color{outcolor}Out[{\color{outcolor}55}]:} [('+861065822892',),
          ('(010)64629112',),
          ('+8601051696505',),
          ('+86-10-60712288',),
          ('68716285;62555813',),
          ('+8613601135725/+861051357212',),
          ('+861051357212',),
          ('+861064428833',),
          ('+861063016688',),
          ('+86-10-64169999',)]
\end{Verbatim}
            
    \subsection{Data Exploration from sql
data}\label{data-exploration-from-sql-data}

    \subsubsection{Size of files}\label{size-of-files}

    \begin{Verbatim}[commandchars=\\\{\}]
{\color{incolor}In [{\color{incolor}63}]:} \PY{n+nb}{print} \PY{p}{(}\PY{l+s+s1}{\PYZsq{}}\PY{l+s+s1}{beijing\PYZus{}china.osm:}\PY{l+s+si}{\PYZob{}\PYZcb{}}\PY{l+s+s1}{ MB}\PY{l+s+s1}{\PYZsq{}}\PY{o}{.}\PY{n}{format}\PY{p}{(}\PY{n}{os}\PY{o}{.}\PY{n}{path}\PY{o}{.}\PY{n}{getsize}\PY{p}{(}\PY{l+s+s1}{\PYZsq{}}\PY{l+s+s1}{beijing\PYZus{}china.osm}\PY{l+s+s1}{\PYZsq{}}\PY{p}{)} \PY{o}{\PYZgt{}\PYZgt{}} \PY{l+m+mi}{20}\PY{p}{)}\PY{p}{)}
         \PY{n+nb}{print} \PY{p}{(}\PY{l+s+s1}{\PYZsq{}}\PY{l+s+s1}{p3\PYZus{}project.db:}\PY{l+s+si}{\PYZob{}\PYZcb{}}\PY{l+s+s1}{ MB}\PY{l+s+s1}{\PYZsq{}}\PY{o}{.}\PY{n}{format}\PY{p}{(}\PY{n}{os}\PY{o}{.}\PY{n}{path}\PY{o}{.}\PY{n}{getsize}\PY{p}{(}\PY{l+s+s1}{\PYZsq{}}\PY{l+s+s1}{p3\PYZus{}project.db}\PY{l+s+s1}{\PYZsq{}}\PY{p}{)} \PY{o}{\PYZgt{}\PYZgt{}} \PY{l+m+mi}{20}\PY{p}{)}\PY{p}{)}
         \PY{n+nb}{print} \PY{p}{(}\PY{l+s+s1}{\PYZsq{}}\PY{l+s+s1}{ways.csv:}\PY{l+s+si}{\PYZob{}\PYZcb{}}\PY{l+s+s1}{ MB}\PY{l+s+s1}{\PYZsq{}}\PY{o}{.}\PY{n}{format}\PY{p}{(}\PY{n}{os}\PY{o}{.}\PY{n}{path}\PY{o}{.}\PY{n}{getsize}\PY{p}{(}\PY{l+s+s1}{\PYZsq{}}\PY{l+s+s1}{ways.csv}\PY{l+s+s1}{\PYZsq{}}\PY{p}{)} \PY{o}{\PYZgt{}\PYZgt{}} \PY{l+m+mi}{20}\PY{p}{)}\PY{p}{)}
         \PY{n+nb}{print} \PY{p}{(}\PY{l+s+s1}{\PYZsq{}}\PY{l+s+s1}{ways\PYZus{}nodes:}\PY{l+s+si}{\PYZob{}\PYZcb{}}\PY{l+s+s1}{ MB}\PY{l+s+s1}{\PYZsq{}}\PY{o}{.}\PY{n}{format}\PY{p}{(}\PY{n}{os}\PY{o}{.}\PY{n}{path}\PY{o}{.}\PY{n}{getsize}\PY{p}{(}\PY{l+s+s1}{\PYZsq{}}\PY{l+s+s1}{ways\PYZus{}nodes.csv}\PY{l+s+s1}{\PYZsq{}}\PY{p}{)} \PY{o}{\PYZgt{}\PYZgt{}} \PY{l+m+mi}{20}\PY{p}{)}\PY{p}{)}
         \PY{n+nb}{print} \PY{p}{(}\PY{l+s+s1}{\PYZsq{}}\PY{l+s+s1}{ways\PYZus{}tags:}\PY{l+s+si}{\PYZob{}\PYZcb{}}\PY{l+s+s1}{ MB}\PY{l+s+s1}{\PYZsq{}}\PY{o}{.}\PY{n}{format}\PY{p}{(}\PY{n}{os}\PY{o}{.}\PY{n}{path}\PY{o}{.}\PY{n}{getsize}\PY{p}{(}\PY{l+s+s1}{\PYZsq{}}\PY{l+s+s1}{ways\PYZus{}tags.csv}\PY{l+s+s1}{\PYZsq{}}\PY{p}{)} \PY{o}{\PYZgt{}\PYZgt{}} \PY{l+m+mi}{20}\PY{p}{)}\PY{p}{)}
         \PY{n+nb}{print}\PY{p}{(}\PY{l+s+s1}{\PYZsq{}}\PY{l+s+s1}{nodes\PYZus{}tags:}\PY{l+s+si}{\PYZob{}\PYZcb{}}\PY{l+s+s1}{ MB}\PY{l+s+s1}{\PYZsq{}}\PY{o}{.}\PY{n}{format}\PY{p}{(}\PY{n}{os}\PY{o}{.}\PY{n}{path}\PY{o}{.}\PY{n}{getsize}\PY{p}{(}\PY{l+s+s1}{\PYZsq{}}\PY{l+s+s1}{nodes\PYZus{}tags.csv}\PY{l+s+s1}{\PYZsq{}}\PY{p}{)} \PY{o}{\PYZgt{}\PYZgt{}} \PY{l+m+mi}{20}\PY{p}{)}\PY{p}{)}
         \PY{n+nb}{print} \PY{p}{(}\PY{l+s+s1}{\PYZsq{}}\PY{l+s+s1}{nodes:}\PY{l+s+si}{\PYZob{}\PYZcb{}}\PY{l+s+s1}{ MB}\PY{l+s+s1}{\PYZsq{}}\PY{o}{.}\PY{n}{format}\PY{p}{(}\PY{n}{os}\PY{o}{.}\PY{n}{path}\PY{o}{.}\PY{n}{getsize}\PY{p}{(}\PY{l+s+s1}{\PYZsq{}}\PY{l+s+s1}{nodes.csv}\PY{l+s+s1}{\PYZsq{}}\PY{p}{)} \PY{o}{\PYZgt{}\PYZgt{}} \PY{l+m+mi}{20}\PY{p}{)}\PY{p}{)}
\end{Verbatim}


    \begin{Verbatim}[commandchars=\\\{\}]
beijing\_china.osm:164 MB
p3\_project.db:87 MB
ways.csv:6 MB
ways\_nodes:21 MB
ways\_tags:7 MB
nodes\_tags:2 MB
nodes:60 MB

    \end{Verbatim}

    \subsubsection{Top 10 contributing users of Beijing
map}\label{top-10-contributing-users-of-beijing-map}

    \begin{Verbatim}[commandchars=\\\{\}]
{\color{incolor}In [{\color{incolor}51}]:} \PY{c+c1}{\PYZsh{}calculate the top 10 contributors for nodes of beijing map}
         \PY{n}{sql\PYZus{}user} \PY{o}{=} \PY{l+s+s1}{\PYZsq{}}\PY{l+s+s1}{select distinct(user), count(*) as num}\PY{l+s+se}{\PYZbs{}}
         \PY{l+s+s1}{            from (select id,user,uid from nodes}\PY{l+s+se}{\PYZbs{}}
         \PY{l+s+s1}{            union }\PY{l+s+se}{\PYZbs{}}
         \PY{l+s+s1}{            select id,user,uid from ways)}\PY{l+s+se}{\PYZbs{}}
         \PY{l+s+s1}{            group by user}\PY{l+s+se}{\PYZbs{}}
         \PY{l+s+s1}{            order by num desc}\PY{l+s+se}{\PYZbs{}}
         \PY{l+s+s1}{            limit 10}\PY{l+s+s1}{\PYZsq{}}
         \PY{n}{c}\PY{o}{.}\PY{n}{execute}\PY{p}{(}\PY{n}{sql\PYZus{}user}\PY{p}{)}\PY{o}{.}\PY{n}{fetchall}\PY{p}{(}\PY{p}{)}
\end{Verbatim}


\begin{Verbatim}[commandchars=\\\{\}]
{\color{outcolor}Out[{\color{outcolor}51}]:} [('Chen Jia', 198317),
          ('R438', 148126),
          ('hanchao', 66845),
          ('ij\_', 52041),
          ('katpatuka', 23674),
          ('m17design', 21795),
          ('Esperanza36', 18399),
          ('nuklearerWintersturm', 17041),
          ('RationalTangle', 14089),
          ('Алекс Мок', 10621)]
\end{Verbatim}
            
    \subsubsection{Hutong numbers in
Beijing}\label{hutong-numbers-in-beijing}

    \begin{Verbatim}[commandchars=\\\{\}]
{\color{incolor}In [{\color{incolor}35}]:} \PY{c+c1}{\PYZsh{} total hutong numbers in Beijing}
         \PY{n}{sql\PYZus{}user} \PY{o}{=} \PY{p}{(}\PY{l+s+s1}{\PYZsq{}}\PY{l+s+s1}{select distinct(value) from nodes\PYZus{}tags where value like }\PY{l+s+s1}{\PYZdq{}}\PY{l+s+s1}{\PYZpc{}}\PY{l+s+s1}{胡同}\PY{l+s+s1}{\PYZdq{}}\PY{l+s+se}{\PYZbs{}}
         \PY{l+s+s1}{            union}\PY{l+s+se}{\PYZbs{}}
         \PY{l+s+s1}{            select distinct(value) from ways\PYZus{}tags where value like }\PY{l+s+s1}{\PYZdq{}}\PY{l+s+s1}{\PYZpc{}}\PY{l+s+s1}{胡同}\PY{l+s+s1}{\PYZdq{}}\PY{l+s+s1}{\PYZsq{}}\PY{p}{)}
         \PY{n+nb}{len}\PY{p}{(}\PY{n}{c}\PY{o}{.}\PY{n}{execute}\PY{p}{(}\PY{n}{sql\PYZus{}user}\PY{p}{)}\PY{o}{.}\PY{n}{fetchall}\PY{p}{(}\PY{p}{)}\PY{p}{)}
\end{Verbatim}


\begin{Verbatim}[commandchars=\\\{\}]
{\color{outcolor}Out[{\color{outcolor}35}]:} 641
\end{Verbatim}
            
    \subsubsection{Source distribution}\label{source-distribution}

    \begin{Verbatim}[commandchars=\\\{\}]
{\color{incolor}In [{\color{incolor}64}]:} \PY{c+c1}{\PYZsh{} source distribution}
         \PY{n}{sql\PYZus{}user} \PY{o}{=} \PY{l+s+s1}{\PYZsq{}}\PY{l+s+s1}{select distinct(value),count(*) as num}\PY{l+s+se}{\PYZbs{}}
         \PY{l+s+s1}{            from ways\PYZus{}tags}\PY{l+s+se}{\PYZbs{}}
         \PY{l+s+s1}{            where key = }\PY{l+s+s1}{\PYZdq{}}\PY{l+s+s1}{source}\PY{l+s+s1}{\PYZdq{}}\PY{l+s+se}{\PYZbs{}}
         \PY{l+s+s1}{            group by value}\PY{l+s+se}{\PYZbs{}}
         \PY{l+s+s1}{            order by num desc}\PY{l+s+s1}{\PYZsq{}}
         \PY{n}{c}\PY{o}{.}\PY{n}{execute}\PY{p}{(}\PY{n}{sql\PYZus{}user}\PY{p}{)}\PY{o}{.}\PY{n}{fetchall}\PY{p}{(}\PY{p}{)}
\end{Verbatim}


\begin{Verbatim}[commandchars=\\\{\}]
{\color{outcolor}Out[{\color{outcolor}64}]:} [('Bing', 3319), ('Yahoo', 1027), ('GPS', 371), ('other', 226), ('Survey', 35)]
\end{Verbatim}
            
    \subsection{Problems encountered in this
project}\label{problems-encountered-in-this-project}

\subsubsection{1. How to deal with Chinese
characters}\label{how-to-deal-with-chinese-characters}

It is inavoidable that some Chinese character is shown in Beijing map.
The problem is that some are in English character but other are in
Chinese, for example, the meaning of** Hutong\textbf{ is }胡同\textbf{,
this case is typical, so I just replace 'hutong' with '胡同'. But it is
tricky to translate Chinese pinying to Chinese, hence I did not
translate pinying to Chines \#\#\# 2. How to translate English into
Chinese I search this question online, and find that a python library
named }goslate** is a common translate api, so I install it just by pip
install. \textbf{but I recognized that goslate can not tranlate pinyin
into Chinese characters}, goslate will do nothing when encountering
strings written by pinyin and English. Lastly, I have to replace those
characters by mapping function. \#\#\# 3. Blank lines between CSV each
rows The first time I try parse OSM file and convert it into CSV file, I
found the \textbf{there are blanks between two data lines} when I open
it by Excel, and then I google this question, and found
\href{https://stackoverflow.com/questions/3348460/csv-file-written-with-python-has-blank-lines-between-each-row/3348664}{this
answer}.\\
\#\#\# 4. Other links used * Element introduction:
\href{https://wiki.openstreetmap.org/wiki/Elements}{wiki}

    \section{Conclusions}\label{conclusions}

    By those project, I found the main problem of this beijing openstreetmap
is its disunity. One way or node could be described by English, Chinese
or pinying, apparently this would increase the difficulty of data
cleaning process. Another problem is phone format which is various in
this case, but for me I am not interesting about the phone, that is why
I did not do much work on this.

Back to the whole process: The key procedures are following: * Parse the
xml file and save the data to a dictionary, the data clean process
occurs in this part, the function is shape\_element() * Transfer data in
dictionary to csv files * Save csv files to sql by pandas to\_sql
modules

 Last but not least, have faith in yourself, I originally should submit
this project long time ago, but I thought it is too difficult for me to
solve, so I started to learn other course and leaving this course no the
half way. After finishing most of course of udacity, I return back to
conquer this course. But now I solve this problems and clean the data
from xml file and then save the data to database successfully, I also
can make it!


    % Add a bibliography block to the postdoc
    
    
    
    \end{document}
